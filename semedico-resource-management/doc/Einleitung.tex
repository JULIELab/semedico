\section{Einleitung}

An dem Language and Information Engineering Lab (JULIE Lab) der Friedrich-Schiller-Universität Jena (FSU Jena) wird seit 2008 an dem Semedico-Projekt gearbeitet. Ziel ist es, eine semantische Suchmaschine zu entwickeln, welche es dem Nutzer ermöglicht schneller und gezielter als mit klassischen Suchmaschinen relevante Veröffentlichungen der Biomedizin zu finden.

%Semedico grenzt sich von anderen Suchmaschinen   

%Eines der wesentlichen Probleme beim wissenschaftlichen Arbeiten ist die Informationssuche. Verschiedenen Studien zufolge verbringt ein Wissenschaftler 50\,\% bis 60\,\% seiner Arbeitszeit mit der Suche nach passenden Informationen TODO:cite. Dieses Problem wird umso größer als die Menge an Daten und Wissen weiterhin exponentiell wächst TODO:cite. Der logische Schluss ist es Methoden zu entwickeln, die es erlauben schnell und gezielt relevante Informationen zu finden. \par

%Eben zu diesem Zweck wird am Language and Information Engineering Lab (JULIE Lab) der Friedrich-Schiller-Universität Jena (FSU Jena) die semantische Suchmaschine Semedico entwickelt.

Semedico baut dabei auf eine Vielzahl von Daten auf, insbesondere aber auf die Medical Subject Headings (MeSH), ein Thesaurus zur Sacherschließung von Texten aus der Medizin und den Biowissenschaften. \par

In der hier vorgelegten Studienarbeit wurde eine Software erarbeitet, die die Verwaltung der verschiedenen Datenquellen des Semedico-Systems vereinheitlicht und vereinfacht. Dies schließt Algorithmen ein, die das Upgrade des MeSH von einer Version zur nächsten vollständig automatisieren. \par 

%\subsection{Semedico}
%\label{sec:semedico}

\minisec{Medical Subject Headings}
Der MeSH ist ein jährlich vom National Library of Medicine (NLM) herausgegebener polyhierarchischer Thesaurus. Er dient zum Katalogisieren von Medien sowie insbesondere zum Indizieren von biomedizinischen Veröffentlichungen. Nach Angaben der offiziellen Webseite werden Artikel von 5\,400 der weltweit führenden biochemischen Journals mit Hilfe des MeSH indiziert \cite{MeSHWeb2012}. \par

Der MeSH enthält Entry Terms, welche jeweils einem Descriptors zugeordnet sind. Zusammen bilden sie eine nach Spezifizität geordnete Polyhierarchie \cite{MeSHWeb2012}. Allgemeine Begriffe wie beispielsweise "`Organisms"' sind folglich weit oben in der Hierarchie zu finden, während spezifischere Begriffe wie zum Beispiel "`Hepatitis Viruses"' diesen untergeordnet sind. \par

Für das Semedico-Projekt dient der MeSH als eine der wesentlichen Datenquellen, da die hierarchische Struktur des MeSH in abgewandelter Form auch als Wissensbasis für Semedico dient. Diese Abwandlung ist insbesondere notwendig, um für Semedico nicht relevante Terme zu entfernen. Zu Veranschaulichung: Der MeSH enthält etwa 200\,000 Entry Terms, welche jeweils einem der über 26\,000 Descriptors zugeordnet sind. Im für Semedio angepassten MeSH finden sich lediglich noch knapp 5\,000 Entry Terms und etwa eben so viele Descriptors.

% \subsubsection{Struktur}
% Es liegen zwei verschiedene Formate vor in denen der MeSH bezogen werden kann: als ASCII-Records sowie als XML-formatierte Records. Tatsächlich enthält die XML-Version einige Daten die in der ASCII-Version nicht vorliegen. \ldots


\minisec{Anmerkung zur Sprache}
Aus Gründen der Lesbarkeit und um Verwirrungen zu vermeiden, werden in dieser Arbeit durchgehend die englischen Originalbezeichnungen von MeSH-spezifischen Begriffen verwendet. \par

Quelltext und Kommentare der entwickelten Software sind vollständig in Englisch. Daher werden Bezeichnungen von Funktionen die eine direkte Entsprechung im Quellcode finden ebenfalls in Englisch bezeichnet. \par 

Abgesehen davon ist die Arbeit vollständig in Deutsch verfasst.